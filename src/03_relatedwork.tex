\section{Related Work}\label{sec:relatedwork}
In order to place the contribution of the paper in context and identify the gap the work is intended to fill, we provide a short literature survey of some existing related studies in the field of EC and topic related papers.\par
The EC paradigm is an actively researched area and has brought forward some papers that we found in particular interesting and useful for our paper. 
Shi et al. \cite{7488250}, Lopez et al. \cite{GarciaLopez:2015:ECV:2831347.2831354}, Varghese et al. \cite{7796149} deal especially with general challenges and opportunities in the field of EC and their multiple application domains. None of the mentioned paper try to solve a problem but rather give a slight overview of different visions and challenges in a generalistic manner. 
In addition to that, Li et al. \cite{7474412} and Ahmed and Rehmani \cite{AHMED201759} explain aspects of Mobile EC and there related fundamental challenges and opportunities and also take privacy and security concerns related to EC into consideration. 
Lopez et al. \cite{GarciaLopez:2015:ECV:2831347.2831354} and Shi et al. \cite{7488250} 
highlight privacy and security as one of the main aspects when focusing on EC.
Architectural aspects were presented by Li et al. \cite{7474412}, Lopez et al. \cite{GarciaLopez:2015:ECV:2831347.2831354} and also Shi et al. \cite{7488250}. All three papers also give an outlook of what visions and challenges has to be considered in the future. Chen et al. \cite{chen2018edge} complement basic architecture aspects with a more industrial view by focusing on an architecture of EC for IoT-based manufacturing environments. The object store service for EC \cite{8014358} gives an insight of how to properly utilize the edge and provide a useful service as well.
Schaefer and Al Yami \cite{al2019fog} capture FC, explain this paradigm in detail and discuss the way how FC complements CC. The role of EC in this context is also being discussed. 

Although visions and challenges of EC have been investigated by many researchers, a lot of past and current research papers focus only on general visions and challenges rather than on an industrial context. Furthermore, new paradigms closely related to EC are not addressed in many articles. 
Hence, our contribution is to describe the vision and challenges of EC not only in a generalistic manner but also in the context of industrial environments. In addition, we briefly present the current state of the art in this field and also bring EC in the context of other emerged paradigms.