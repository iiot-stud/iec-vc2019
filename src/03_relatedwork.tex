\section{Related Work}\label{sec:relatedwork}

In order to place the contribution of the paper in context and identify the gap the work is intended to fill, we provide a short literature survey of some existing related studies in the area of Edge Computing and topic related papers.

The concept of the edge computing paradigm is an actively researched area and has brought forward some papers that we found in particular interesting.
These papers dealt in particular with challenges and opportunities from their point of view \cite{7488250}\cite{GarciaLopez:2015:ECV:2831347.2831354}\cite{7796149}\cite{7474412}\cite{AHMED201759} but none of them were trying to solve a problem but rather gave a slight overview of different visions and challenges.

One of the main aspects of edge computing is privacy but also security which were discussed by \cite{7474412}\cite{GarciaLopez:2015:ECV:2831347.2831354}\cite{AHMED201759}\cite{7488250}.

Architectural aspects were presented in \cite{7474412}\cite{GarciaLopez:2015:ECV:2831347.2831354}\cite{7488250} and gave an overview of what visions and challenges lie ahead.

The object store service for edge computing \cite{8014358} gives an insight of how to properly utilize the edge and provide a useful service as well. 


\sidenote {Responsible: all}
\todo{introduction and short summary}\\
\todo{Feedback from Alex: Existing Work: Survey past work relevant to this paper. Why hasn’t it been solved before (related work)? Or, what’s wrong with previous proposed solutions?}

\todo{Li, H., Shou, G., Hu, Y., \& Guo, Z. (2016). Mobile edge computing: Progress and challenges. In Proceedings - 2016 4th IEEE International Conference on Mobile Cloud Computing, Services, and Engineering MobileCloud 2016 (pp. 83–84). https://doi.org/10.1109/MobileCloud.2016.16\\
\\
- Varghese, B., Wang, N., Barbhuiya, S., Kilpatrick, P., \& Nikolopoulos, D. S. (2016). Challenges and Opportunities in Edge Computing. Ieeexplore.Ieee.Org. https://doi.org/10.1109/SmartCloud.2016.18\\
\\
- Lopez, P. G., Montresor, A., Epema, D., Datta, A., Higashino, T., Iamnitchi, A., … Riviere, E. (2005). Edge-centric Computing: Vision and Challenges. ACM SIGCOMM Computer Communication Review, 45(5), 37–45. https://doi.org/10.1145/2831347.2831354\\
\\
- Ahmed, E., \& Rehmani, M. H. (2017). Mobile Edge Computing: Opportunities, solutions, and challenges. Future Generation Computer Systems, 70, 59–63. https://doi.org/10.1016/j.future.2016.09.015\\
\\
- Shi, W., Cao, J., Zhang, Q., Li, Y., \& Xu, L. (2016). Edge Computing: Vision and Challenges. IEEE Internet of Things Journal, 3(5). https://doi.org/10.1109/JIOT.2016.2579198\\
\\
- Ceri, S., Braga, D., Corcoglioniti, F., Grossniklaus, M., \& Vadacca, S. (2010). Search computing challenges and directions. In Lecture Notes in Computer Science (including subseries Lecture Notes in Artificial Intelligence and Lecture Notes in Bioinformatics) (Vol. 6348 LNCS, pp. 1–5). https://doi.org/10.1007/978-3-642-16092-9\_1}\\

\todo{add further references/conference paper and describe further related work, if necessary}

