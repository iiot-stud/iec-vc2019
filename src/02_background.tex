\section{Theoretical Background}\label{sec:background}
 
 \sidenote{Responsible: Lukas}
\todo{- Definition: Industrial Edge Computing --> What is (Industrial) Edge Computing, current vision/idea in the contect of industrial edge computing}\\

In today's information technology age, data is a main commodity that generates an important contribution for data-driven businesses in industrial companies. Data are mainly produced at the edge of a network, primarily by edge devices (e.g. smartphones). In particular, the forecast of 50 billion devices which will be interconnected in the context of the Consumer and Industrial Internet of Things (IoT) by 2020 will lead to a further massive generation of data. These data will be made usable through applications developed at the edge of a network. Nowadays the data used by these applications for processing is mostly stored, analyzed and processed in data centers in a cloud based environment.

\subsection{Differentiation Cloud Computing vs. Edge Computing vs. Fog Computing}
\todo{make/use a figure}

\subsection{Cloud Computing paradigm to Edge Computing paradigm}
\todo{- describe development from cloud computing paradigm to edge computing paradigm incl. reasons}

\subsection{Standards and initiatives}
\todo{- check for initiatives/standards/organizations in edge computing}\\

First research results:\\
Main existing Edge Computing reference architectures:\\
- Edge Computing Consortium (https://ecconsortium.eu/)\\ 
- FAR-Edge Project (https://www.edge4industry.eu/knowledge-base/articles/far-edge-reference-architecture/)\\
- Industrial Internet Consortium for Industry 4.0\\

Goal: (Common) Edge Computing Reference Architecture