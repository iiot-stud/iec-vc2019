\section{Theoretical Background}\label{sec:background}

In today's information technology age, data is a main commodity that generates an important contribution for data-driven businesses in industrial companies. Data are mainly produced at the edge of a network, primarily by edge devices (e.g. smartphones). In particular, the forecast of 50 billion devices which will be interconnected in the context of the Consumer and Industrial Internet of Things (IoT) by 2020 will lead to a further massive generation of data. These data will be made usable through applications developed at the edge of a network. Nowadays the data used by these applications for processing is mostly stored, analyzed and processed in data centers in a cloud based environment. The massive and rapid increase in the volume of data, resulting in particular from the steadily increasing number of connected IoT devices and applications, leads to massive technical challenges with regard to the amount of data, which will be transferred to the cloud. In the age of digitalization, time-sensitive and location-aware applications (e.g. real-time manufacturing/monitoring, remote assistance) are becoming increasingly important for industrial companies.\todo{insert reference of paper}\par
In addition to physical bandwidth constraints and the lack of scalability of the data produced by such applications, the transfer of data to the cloud, which is far away from a network topological standpoint, cannot meet the low-latency requirements of such applications. The cloud is also not designed to provide location-aware services.\todo{Think about considering aspect: privacy concerns as well?)} In dealing with these issues, a modern computing paradigm is needed that acts closer to connected devices. Edge computing is such a computing paradigm, which shifts the possibilities of computing power, storage, network connection and data management from the cloud closer to the edge of a network, meaning in close proximity to devices, sensors and users.
\todo{insert reference of paper} This leads in particular in the industrial IoT context to substantial benefits in the area of agility, autonomy and real-time processing of data. For an intelligent factory of the future, this is an important prerequisite to open up for innovative and value-adding opportunities.\par 
In the course of the discussion of this vision, several paradigms have emerged which have been established in cooperation with various industrial consortia and research institutions. In order to get an overview of the most important paradigms and to provide the reader with a clear understanding, edge computing paradigm will be differentiated from fog computing paradigm. Also the technical evolution from cloud computing paradigm to edge computing paradigm will be discussed in the next sub chapter.

\subsection{Differentiation computing paradigm: Edge Computing vs. Fog Computing}
\sidenote{Responsible: Lukas}
\todo{figure or explain the differences, take also Cloud Computing into consideration.}\\
\todo{Cloud Computing paradigm to Edge Computing paradigm: No own sub section - in 2.1: describe development from cloud computing paradigm to edge computing paradigm incl. reasons}
\todo{consistent link to subchapter 2.2}

\subsection{Standards and initiatives}
\todo{- check for initiatives/standards/organizations in edge computing}\\
First research results:\\
Main existing Edge Computing reference architectures:\\
- Edge Computing Consortium (https://ecconsortium.eu/)\\ 
- FAR-Edge Project (https://www.edge4industry.eu/knowledge-base/articles/far-edge-reference-architecture/)\\
- Industrial Internet Consortium for Industry 4.0\\

Goal: (Common) Edge Computing Reference Architecture