\section{Industrial Edge Computing}\label{sec:background}
In today's technology and information era, data is a main commodity that generates an important contribution for data-driven businesses in industrial companies. Data is mainly produced at the edge of a network, primarily by edge devices (e.g. smartphones, sensors). In particular, the forecast of 50 billion devices which will be interconnected in the context of the Consumer and Industrial IoT by 2020 will lead to a further massive generation of data \cite{7488250}. This data will be made usable through applications developed at the edge of a network. Nowadays the data used by these applications for processing is mostly stored, analyzed and processed in data centers in a cloud based environment. Especially in an industrial context this can lead to multiple issues and challenges which are addressed by proposals and research projects under the EC paradigm.\par
In the following sub-chapters, the technological background, the underlying problem and the differentiation to other existing computing paradigms will be presented as well as various initiatives and industrial consortia that are currently involved in working on a EC reference architecture and common standards.

\subsection{Background and Problem}\label{2.1}
The massive and rapid increase in the volume of data, resulting in particular from the steadily increasing number of connected IoT devices and applications, leads to technical challenges with regard to the amount of data, which will be transferred to the cloud. In the age of digitalization, time-sensitive and location-aware applications (e.g. real-time manufacturing/monitoring, remote assistance) are becoming increasingly important for industrial companies \cite{yousefpour2019all}.\par
In addition to physical bandwidth constraints and the lack of scalability of the data produced by such applications, transfer of the data to the cloud, which is usually far away from a network topological standpoint, cannot meet the low-latency requirements of such applications. The cloud is also not designed to provide location-aware services. By also considering privacy concerns, storage and processing data in the could may not be possible due to regulations and industry internal directives \cite{yousefpour2019all}.\par
To deal with these issues, a modern computing paradigm is needed that acts closer to connected devices. EC is such a paradigm, which shifts the possibilities of computing power, storage, network connection and data management from the cloud closer to the edge of a network, meaning in close proximity to devices, sensors and users. This leads in particular in the Industrial IoT context to substantial benefits in the area of agility, autonomy and real-time processing of data. For an intelligent factory of the future, this is an important prerequisite to open up for innovative and value-adding opportunities \cite{chen2018edge}.\par 
In the course of the discussion of this vision, several paradigms have emerged which have been established in cooperation with various industrial consortia and research institutions. In order to obtain an overview of the most important paradigms in the context of this topic and to provide the reader with a clear understanding, the EC and the FC paradigm will be explained and differentiated from each other. Also the underlying evolution in the industrial context from CC to EC will be discussed.

\subsection{Differentiation of Related Computing Paradigms}
In the course of the last decade, the CC paradigm opened up new opportunities in data computing and availability. Particularly in the environment of enterprise IT infrastructure (mainstream servers and desktop computers), CC created enormous added value for industrial companies because of the possibility of obtaining fluctuating and unpredictable computational demand at low cost (e.g. pay-as-you-go principle) and, on top of that, making the data to be processed available everywhere and for anybody desired.
Due to the increase in the variety and number of end devices through the availability of wearable technology and in particular the strong spread of IoT as an enabler for smart environments (e.g. smart manufacturing), requirements and the scope of computational devices used at the edge of a network have been changed in a way as mentioned detailed in subchapter \ref{2.1}. These aspects cannot be satisfied by only taking CC into consideration \cite{baktir2017can}.
The intention and specification of the mentioned paradigms is different. Table \ref{table:1} compares briefly selected features of CC and EC. Nevertheless, both technologies should not be seen as interchangeable, but rather as interrelated.
For this reason, various innovative concepts such as micro datacenter, mobile edges and cloudlets have been introduced in various research projects in addition to CC \cite{7488250}.
All mentioned concepts have different characteristics but refer to the overall EC paradigm.

\begin{table}[h!] 
\centering
\begin{tabular}{ |p{2.7cm}|p{4.6cm}|p{4.4cm}|  }
 \hline
 Features & Cloud Computing & Edge Computing \\ [0.5ex] 
 \hline\hline
Latency & High & Low \\
Server Location & Anywhere (within a network) & At the edge \\
Distribution & Centralized & Distributed \\
Mobility Support & Low & High \\
Application Scope & Higher computational power & Lower latency \\
User Device & Only computers, mobile devices & Mobile, smart devices \\
 \hline
\end{tabular}
\caption{Scope of Cloud and Edge Computing \cite{baktir2017can}}
\label{table:1}
\end{table}
Besides the CC paradigm and in addition to the EC paradigm, the scientific literature also provides another computing paradigm - FC, which is becoming increasingly important in the course of current researches.

EC and FC have the same intention to move storage and computational power to the edge of a network, close to the end nodes. This shows that both paradigms are interconnected, but not identical. According to the OpenFog Consortium, FC differs from EC by its characteristics. Fog is hierarchically structured and enables computing, networking, storage, control and acceleration anywhere between cloud and things. EC, on the other hand, tends to focus only to computing at the edge \cite{yousefpour2019all}. According to an IEEE tutorial article, FC is composed end-to-end of cloud, core, metro, edge, clients and things with the goal to realize a continuum of computing services from the cloud to the things, whereas EC considers only the network edges as an isolated computing platform \cite{chiang2017clarifying}. Fog also aims to establish a horizontal platform that provides different FC functions for different industries and application domains \cite{yousefpour2019all}.
Recently, a new paradigm has been introduced due to the increasing popularity of IoT devices. The Mist Computing paradigm can be described as "extreme edge" in the context of IoT, because it describes dispersed computing at the extreme edges, i.e. on the IoT peripherals themselves, and not the immediate first hop from the IoT devices (EC) \cite{yousefpour2019all}.

In the context of industrial EC some initiatives and initial standardization approaches currently exist. Some of them are described in the following sub-chapter. 

\subsection{Standards and Initiatives}
As of today various industrial and commercial organizations, academic institutions and governments actively research and promote the standardization and industrialization of EC.
In the academic field, EC gained importance at the end of 2016 in the course of the first IEEE/ACM Symposium on Edge Computing\footnote{\url{http://acm-ieee-sec.org/2016/}}, which focused in particular on research orientations and the key value of EC.
The first standardization approaches were launched with the publication of the International Electrotechnical Commission (IEC) white paper "Vertical Edge Intelligence"\footnote{\url{https://www.iec.ch/whitepaper/edgeintelligence/}} in 2017. This paper focuses on the value of EC for vertical industries such as manufacturing.
With the establishment of a research group within the framework of ISO/IEC JTC 1/SC 41\footnote{\url{https://www.iso.org/committee/6483279.html}} at the beginning of 2018 and the associated project ISO/IEC TR 30164 ED1, a standardization of EC is to be pushed forward.\par
In August 2017, the Industrial Internet Consortium (IIC) and the Edge Computing Consortium (ECC) joined forces to establish a common reference architecture \cite{IICEC2}.
With the Industrial Internet Reference Architecture (IIRA), a reference architecture model was established, which is the basis for analysis of Industrial IoT systems and also the basis for analyzing edge nodes.
In addition to the IIRA, there exists a Reference Architecture Model for Edge Computing (RAMEC), which was initiated by the planned Edge Computing Consortium Europe (ECCE) and will be further developed and described. In addition to the specification of the ECCE RAMEC, the main focus will be on the development of reference technology stacks as well as the synchronization with other related initiatives and standards \cite{ECCE}.
The Factory Automation Edge Computing Operating System Reference Implementation project with the FAR-Edge reference architecture and the associated FAR-EDGE platform aims at a multi-sided ecosystem for industrial automation based on EC that uses blockchain technologies as a synchronization tool between the information models and the current status of the factory \cite{E4I}.
Another initiative is known as the Multi-Access Edge Computing (MEC) initiative, which was launched by the European Telecommunication Standards Institute (ETSI) and provides a standardized open environment for various stakeholders via a platform \cite{ETSI}.
In early 2019, the two largest international consortia IIC and OpenFog Consortium teamed up to form the largest international consortium concentrating on Industrial IoT, FC and EC \cite{IICOF}. The OpenFog Consortium also provides a reference architecture focusing mainly on FC, but also considers EC as part of FC. The new formed consortium focuses on industry guidance and best practices for FC and EC \cite{IICEC}.