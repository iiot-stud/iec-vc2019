\section{Related Work}\label{sec:relatedwork}

In order to place the contribution of the paper in context
  and identify the gap the work is intended to fill,
  we provide a short literature survey.
\sidenote {Responsible: all}
\todo{introduction and short summary}\\
\todo{Li, H., Shou, G., Hu, Y., \& Guo, Z. (2016). Mobile edge computing: Progress and challenges. In Proceedings - 2016 4th IEEE International Conference on Mobile Cloud Computing, Services, and Engineering MobileCloud 2016 (pp. 83–84). https://doi.org/10.1109/MobileCloud.2016.16\\
\\
- Varghese, B., Wang, N., Barbhuiya, S., Kilpatrick, P., \& Nikolopoulos, D. S. (2016). Challenges and Opportunities in Edge Computing. Ieeexplore.Ieee.Org. https://doi.org/10.1109/SmartCloud.2016.18\\
\\
- Lopez, P. G., Montresor, A., Epema, D., Datta, A., Higashino, T., Iamnitchi, A., … Riviere, E. (2005). Edge-centric Computing: Vision and Challenges. ACM SIGCOMM Computer Communication Review, 45(5), 37–45. https://doi.org/10.1145/2831347.2831354\\
\\
- Ahmed, E., \& Rehmani, M. H. (2017). Mobile Edge Computing: Opportunities, solutions, and challenges. Future Generation Computer Systems, 70, 59–63. https://doi.org/10.1016/j.future.2016.09.015\\
\\
- Shi, W., Cao, J., Zhang, Q., Li, Y., \& Xu, L. (2016). Edge Computing: Vision and Challenges. IEEE Internet of Things Journal, 3(5). https://doi.org/10.1109/JIOT.2016.2579198\\
\\
- Ceri, S., Braga, D., Corcoglioniti, F., Grossniklaus, M., \& Vadacca, S. (2010). Search computing challenges and directions. In Lecture Notes in Computer Science (including subseries Lecture Notes in Artificial Intelligence and Lecture Notes in Bioinformatics) (Vol. 6348 LNCS, pp. 1–5). https://doi.org/10.1007/978-3-642-16092-9\_1}\\

\todo{add further references/conference paper and describe further related work, if necessary}

