\section{Introduction}\label{sec:introduction}
In the initial stage of IT system development, data processing had taken place in mainframe computers, subsequently transferred to personal computers with their popularization, and then again centralized - in the cloud. Edge computing, the next phase of the cycle, is the return of the concept of distributed architecture, but in the new version. This paradigm is closely connected to the Internet of Things (IoT) and it's based on the processing, and often also on the storage of data by end devices, controllers, or micro-data centers operating near to them, i.e. on the edge of the network.\par 
Like every architecture, also edge computing has its advantages and disadvantages. Certainly, this paradigm has great potential in the context of undertaking various digital initiatives especially in the context of Industrial IoT. However, organizations should follow an accepted and well-thought strategy when adapting edge computing, because the current challenges and requirements like scalability, security, privacy and others are complex and not to be underestimated.\par
Edge computing is not a completely new concept, but virtually none scientific paper focus completely on the challenges with which companies have to face when implementing this approach. Therefore, we decided to focus on gathering in one paper the main requirements that need to be considered when focusing on edge computing. \par
We submit a detailed background information on what industrial edge computing is, a differentiation between edge computing and fog computing, available standards and initiatives and the challenges that are currently facing edge computing based on a review of applicable literature and conference papers. In order to present satisfying results, or an up-to-date vision and review of the main challenges related to edge computing, we focused on a detailed research based on a comprehensive literature review and own representations of the results. \par
Our analysis shows that even with the enormous benefits that edge computing could deliver for industrial applications, the technological framework and the definition of common standards still take time to mature. Moreover, the unrestricted adaptability to real world industrial applications is not fully sophisticated yet.\par 
This work could be a good start for a further investigation of the edge computing concept in order to answer many questions that arise regarding to this paradigm. One of the main issues is considering whether edge computing can be seen as a cloud computing extension, or maybe even one would be tempted to say that edge computing is able to completely replace cloud computing.\par
At the beginning, in the "Theoretical Background" section, we comprehensively explain the edge computing concept and also present the differences between edge, cloud and fog computing. In the next section, we submit a brief literature overview, on the basis of which we made our analysis. In the main section we present visions and challenges related to the implementation of edge computing approach. Afterwards, we present the current state of the art in this topic and finally, you can find our summary and conclusions that we came to while working on this issue.\par
