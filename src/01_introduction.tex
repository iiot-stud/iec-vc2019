\section{Introduction}\label{sec:introduction}
In the initial stage of IT system development, data processing had taken place in mainframe computers, subsequently transferred to personal computers with their popularization, and then again centralized - in the cloud. Edge computing, the next step in development, is the return of the concept of distributed architecture, but in a new way with different specifications. This paradigm is closely connected to the Internet of Things (IoT) and is based on the processing, and often also on the storage of data by end devices, controllers, or micro-data centers operating near to them, i.e. on the edge of the network.\par 
Like every architecture, also edge computing has its advantages and disadvantages. Certainly, this paradigm has great potential in the context of undertaking various digital initiatives especially in the context of Industrial IoT. However, organizations should follow an accepted and well-thought strategy when adapting edge computing, because the current challenges and requirements like scalability, security, privacy and others are complex and not to be underestimated, because the interaction of these aspects in the framework of edge computing is in the focus of current research.\par
Edge computing is not a completely new concept, but virtually none scientific paper focus completely on the challenges with which industry have to face when implementing this approach. Therefore, we decided to focus on gathering in one paper the main requirements that need to be considered when focusing on edge computing. \par
We provide detailed background information on what industrial edge computing is, a differentiation between edge computing and fog computing, available standards and initiatives and the challenges that are currently facing edge computing based on a review of applicable literature and conference papers. In order to present satisfying results in the way of an up-to-date vision and review of the main challenges related to edge computing, we provide a detailed summarization based on own representations of the results. \par
Our analysis shows that even with the enormous benefits that edge computing could deliver for industrial applications, the technological framework and the definition of common standards still take time to mature. Moreover, the unrestricted adaptability to real world industrial applications is not fully sophisticated yet.\par 
This work can be seen as a starting point for further research on this topic, that will contribute to answering the open questions that exist in the context of edge computing. One of the main issues is considering whether edge computing can be seen as a cloud computing extension, or even has the potential to replace cloud computing.\par
At the beginning, in the "Theoretical Background" section, we comprehensively explain the edge computing concept and also present the differences between edge, cloud and fog computing. In the next section, we submit a brief literature overview, on the basis of which we made our analysis. Followed by a presentation of visions and challenges related to the implementation of the edge computing paradigm. Afterwards, we present the current state of the art to this topic. Finally, we come up with a summary and conclusion based on our research results.\par
