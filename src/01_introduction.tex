\section{Introduction}\label{sec:introduction}
In the initial stage of IT system development, data processing had taken place in mainframe computers, subsequently transferred to personal computers with their popularization, and then again centralized - in the cloud. EC, the next step in development, is the return of the concept of distributed architecture, but in a new way with different specifications. This paradigm is closely connected to the IoT and is based on the processing, and often also on the storage of data by end devices, controllers or micro-data centers operating near to them, i.e. at the edge of a network.\par 
Like every architecture, also EC has its advantages and disadvantages. Certainly, this paradigm has great potential in the context of undertaking various digital initiatives especially in the context of Industrial IoT. However, organizations should follow an accepted and well-thought strategy when adapting EC, because the current challenges and requirements such as scalability, security, privacy and others are complex, not fully investigated and therefore not to be underestimated. The interaction of these aspects in the framework of EC is in the focus of current research initiatives.\par
EC is not a completely new paradigm, but virtually no scientific paper focuses completely on the challenges which industries have to face when implementing this approach. Therefore, the main focus of this paper is to show the latest visions, challenges, and related requirements that need to be considered when focusing on EC. In addition to that the current state of the art and trends in this domain will be presented. \par
Starting point of this paper is to provide reader with detailed background information on what industrial EC is, a differentiation between EC and FC as well as introduce available standards and initiatives. Followed by challenges that are currently facing EC based on a review of applicable literature and conference papers. In order to present satisfying results in a way of an up-to-date vision and main challenges related to EC, a detailed summary based on own representations of the results is the outcome of this work. \par
Our analysis shows that even with the enormous benefits that EC could deliver for industrial applications, the technological framework and the definition of common standards still take time to mature. Moreover, the unrestricted adaptability to real world industrial applications is not fully sophisticated yet.\par 
This work can be seen as a starting point for further research on this topic that will contribute to answer the open questions that exist in the context of EC. One of the main considerations is whether the EC paradigm is, in the industrial context, only an extension to CC or whether it has to be interconnected to the FC paradigm to create an efficient solution.\par
First, in the chapter "Industrial Edge Computing", the basics of EC as well as the differences between Cloud, Edge and Fog Computing will be comprehensively explained. Also already known initiatives and basic standardization approaches in the field of EC are part of this chapter. In the next section, we submit a brief literature overview on the basis of which we made our analysis. Followed by a presentation of current visions and challenges related to the implementation of the EC paradigm. Afterwards, we present the current state of the art of this topic. Finally, we come up with a conclusion as well as a brief outlook based on our research results.\par