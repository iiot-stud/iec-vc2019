\section{Introduction}\label{sec:introduction}
In the initial stage of IT system development, data processing had taken place in mainframe computers, subsequently transferred to personal computers with their popularization, and then again centralized - in the cloud. Edge Computing (EC), the next step in development, is the return of the concept of distributed architecture, but in a new way with different specifications. This paradigm is closely connected to the Internet of Things (IoT) and is based on the processing, and often also on the storage of data by end devices, controllers, or micro-data centers operating near to them, i.e. at the edge of a network.\par 
Like every architecture, also EC has its advantages and disadvantages. Certainly, this paradigm has great potential in the context of undertaking various digital initiatives especially in the context of industrial IoT. However, organizations should follow an accepted and well-thought strategy when adapting EC, because the current challenges and requirements such as scalability, security, privacy and others are complex, not fully investigated and therefore not to be underestimated. The interaction of these aspects in the framework of EC is in the focus of current research initiatives.\par
EC is not a completely new paradigm, but virtually none scientific paper focuses completely on the challenges which industry have to be faced with when implementing this approach. Therefore, the main focus of this paper is to show latest visions and challenges and related requirements that need to be considered when focusing on EC. In addition to that the current state of research and trends in that domain will be presented. \par
Starting point of this paper is to provide the reader with detailed background information on what industrial EC is, a differentiation between EC and fog computing as well as introduce to available standards and initiatives. Followed by challenges that are currently facing EC based on a review of applicable literature and conference papers. In order to present satisfying results in the way of an up-to-date vision and main challenges related to EC, a detailed summary based on own representations of the results is the outcome of this work. \par
Our analysis shows that even with the enormous benefits that edge computing could deliver for industrial applications, the technological framework and the definition of common standards still take time to mature. Moreover, the unrestricted adaptability to real world industrial applications is not fully sophisticated yet.\par 
This work can be seen as a starting point for further research on this topic, that will contribute to answer the open questions that exist in the context of edge computing. One of the main issues is considering whether edge computing can be seen as a cloud computing extension, or even has the potential to replace cloud computing.\par
First, in the chapter "Industrial Edge Computing", the basics of edge computing as well as  the differences between cloud, edge and fog computing will be comprehensively explained. In the next section, we submit a brief literature overview, on the basis of which we made our analysis. Followed by a presentation of visions and challenges related to the implementation of the edge computing paradigm. Afterwards, we present the current state of the art of this topic. Finally, we come up with a summary and conclusion as well as a brief outlook based on our research results.\par
\todo{review after finishing paper}