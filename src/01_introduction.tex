\section{Introduction}\label{sec:introduction}
In the initial stage of IT system development, data processing had taken place in mainframe computers, subsequently transferred to personal computers with their popularization, and then again centralized - in the cloud. Edge computing, the next phase of the cycle, is the return of the concept of distributed architecture, but in the new version.\par
In contrast to the mainframe computer era, at the time of popularization of personal computer, data processing has been distributed into end-user devices - PCs, laptops, tablets, and more recently to some extent as smartphones. When the cloud appeared, data processing was re-centralized, and the end-user devices function was reduced to delivering and receiving information processed in the central cloud system. Responsibility for calculation moves systematically towards the cloud or remote data centers, and some specialists predict that soon computer applications will mainly use data and processing power made available in the clouds. Thus, the computer will be "degraded" to the terminal.\par
Edge computing is a step in the opposite direction. Although in a different form, we return to the idea of distributed computing, where data is stored, processed, and even analyzed directly in end-user devices, controllers or micro data centers located in their immediate vicinity.\par
This study covers detailed background information on what industrial edge computing is, a differentiation between cloud computing and edge computing, available standards and initiatives, a review of applicable literature, and the challenges that are currently facing edge computing.\par
 
 
