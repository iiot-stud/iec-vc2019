\section{Conclusion/Summary and Outlook}\label{sec:conclusions}

\todo{Conclusion or Summary? Important Outlook}
\todo{- summary of our paper, - potential outcome, Result: the scientific surplus value, - Outlook: Basis for what kind of next work?}
\todo{more detailed, have to consider the outcomes of our paper, maybe bring pyramid in here to explain the current state but also the future work in combination with upcoming standards and initiatives --> that should be the basis of next work and may also be the scientific surplus value}

Edge computing is an architectural style for distributed systems. It's gaining popularity for many reasons, among others by reducing the costs of data transfer to the cloud or its own environment, reducing response time or enabling security - avoiding the transmission and storage of unsecured raw data in the cloud. Edge computing has its indisputable advantages, but it also has disadvantages. For this reason, it should be used deliberately in situations where the advantages will outweigh the drawbacks. In this paper we've presented in detail current challenges and visions related to edge computing. The main visions for the future were determined to be the achievement of low latency, the reduction of bandwidth, and the notion of edge computing as a service. However, visions usually come along with challenges and those were identified as architectural, privacy, and security issues such as communication, scalability, and legal constraints. This work may be the basis for further research on the edge computing paradigm to establish dependencies and connections between cloud, edge and fog computing.

To put it more concretely, while analyzing the provided papers and reading other related work it became clear that a glue layer in between cloud and edge is currently missing. This is valid for some approaches, while others briefly considered it as an option but did not pursue it further.
Therefore we see fog computing as a paradigm that can fill the missing piece and be the glue layer between cloud and edge.
The fog would be able to connect edge and cloud together to provide the benefits of the other two paradigms by providing high computing power in the cloud for intensive calculations while still being able to deliver content with low latency from either fog resources or edge resources.


\todo{in the end checK: abbroviations, spelling, capital and small letter, all references included and referenced}