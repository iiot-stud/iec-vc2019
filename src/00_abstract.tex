The Edge Computing (EC) paradigm is related to the Internet of Things (IoT) and is based on the idea of processing and storing data by end devices operating in close proximity to the computational power at the edge of a network. Within the IoT context, challenging requirements have to be met when applying this approach. Currently, most research and conference papers related to EC focus either on the advantages or disadvantages of this concept, which is why we present a precise list of requirements related to this approach. We carry out a detailed research based on a literature review and own representations of the information collected. The core contribution is to explain the main idea as well as to summarize the challenges that organizations are faced with when focusing on EC. Based on this work, EC paradigm can be investigated to answer the raising questions of whether it can be seen as an extension to cloud computing or even has the potential to replace it.