The Edge Computing (EC) paradigm is related to the Internet of Things (IoT) and is based on the idea of processing and storing data near devices, sensors and users at the edge of a network. Within the industrial context, challenging requirements have to be met when applying this approach. Many research papers related to EC focus either on the advantages or disadvantages, which is why we present a summary of visions and challenges and introduce the current state of the art related to this paradigm. We carry out a detailed research based on a literature review and own representations of the information collected. The core contribution is to provide clear visions and challenges in the industrial context that organizations are faced with. Based on this, it can be investigated further whether EC can only be seen as extension to Cloud Computing (CC) or whether it has to be interconnected to the Fog Computing (FC) paradigm.