Edge computing is not a new phenomenon. This concept has already appeared some time ago and is closely connected to the Internet of Things (IoT). It is highly related to the storage, processing and computation of data by end devices, controllers, or micro-data centers operating on the edge of the network. The purpose of this article is to present a vision and review the main challenges of edge computing since every concept has its advantages and disadvantages. In order to explain the main idea as well as to summarize the challenges that organizations are faced with when focusing on edge computing, we carry out a comprehensive research based on a literature review and own representations of the information collected. Our analysis shows that even with the enormous benefits that edge computing could have in industrial applications, the technological framework and the definition of common standards still take time to mature. Moreover the full adaptability to real world industrial applications is not fully sophisticated yet. Certainly, edge computing has great potential in the context of undertaking various digital initiatives especially in the context of IoT. However, organizations should follow an accepted and well-thought strategy when adapting edge computing, because the challenges and the associated actions are complex and not to be underestimated.